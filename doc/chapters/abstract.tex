%! Author = silva
%! Date = 6/12/2022

\begin{center}
    \bfseries
    "Ubersicht
    \normalfont
\end{center}
Verschiedene Arten von Bakterien beginnen, Resistenzen gegen einige Arten von Antibiotika zu zeigen, und Studien haben eine Zunahme von Antibiotikaresistenz-assoziierten Krankheiten im Zusammenhang mit diesen nicht erregerspezifischen Therapien gezeigt.
Diese Entwicklungen beg"unstigen den R"uckgriff auf erregerspezifische Ans"atze, erfordern jedoch einen hohen manuellen Aufwand und sind dadurch arbeits- und zeitintensiv.
In dieser Hinsicht k"onnten Mittel zum Nachweis des Bakterientyps durch Analyse mikroskopischer Bilder von Fl"ussigkeitsproben unter Verwendung moderner Bildverarbeitungs- und maschineller Lernwerkzeuge die unerschwinglichen Kosten manueller Methoden vermeiden.
Daher wollen diese Strategien untersucht werden, um das Problem der automatischen Objekterkennung in mikroskopischen Bildern von Fl"ussigkeitsproben anzugehen.

Translate this page



\vspace{5.0cm}

\begin{center}
    \bfseries
    Abstract
    \normalfont

\end{center}

    Various types of bacteria are starting to show resistance against some types of antibiotics and studies have shown an increase in antibiotic-resistance-associated diseases linked to these non-pathogen-specific therapies.
    These developments encourage reverting to pathogen-specific approaches however It requires a high amount of manual work and is thereby laborious and time-consuming.
    In this regard, means to detect the type of bacteria by analysing microscopic images of fluid samples using modern image processing and machine learning tools could avoid the prohibitive costs of manual methods.
    Therefore, want to investigate these strategies to address the problem of automatic object detection in microscopic images of fluid samples.